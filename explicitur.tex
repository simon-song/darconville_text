\chapter*{Explicitur}
\addcontentsline{toc}{chapter}{Explicitur}%

~~THIS IS A STORY of murder, which, as an act, is as apt to characterize
deliverance as it is to corroborate death. There are certain elemental
emotions that touch upon powers other and larger than our own discrete
wishes might allow---for every consciousness is continuous with a wider
self open to the hidden processes and unseen regions created in the soul
by the very nature of an \emph{opposite} effort---and while, taken
together, each may prove the other simply by contrast, considered
separately neither may admit of various shades in the law of whichever
whole it finds reigning at the time. That which produces effects within
one reality creates another reality itself. I am thinking, specifically,
of love and hate.

~~We cannot distinguish, perhaps, natural from supernatural effects, nor
among either know which are favors of God and which are counterfeit
operations of the Devil. Who, furthermore, can speak of the incubations
of motives? And of love and hate? Are they not too often, in spite of
the comparative chaos within us, generally taken to be little more than
a set of titles obtained by the mere mechanical manipulation of
antonyms? I have no aspiration here to reclaim mystery and paradox from
whatever territory they might inhabit, for there is, indeed, often a
killing in a kiss, a mercy in the slap that heats your face.

~~There is, nevertheless, a particular poverty in those alloplasts who,
addressing tragedy, seek to subdistinguish motives beyond those we have
best, because nearest, at hand, and so it is with love and hate---
emotions upon whose necks, whether wrung or wreathed, may be found the
oldest fingerprints of man. A simple truth intrudes: the basic instincts
of every man to every man are known. But who knows when or where or how?
For the answers to such questions, summon Augurello, your personal
jurisconsult and theological wiseacre, to teach you about primal reality
and then to dispel those complexities and cabals you crouch behind in
this sad, psychiatric century you call your own. It is the
\emph{anti}-labyrinths of the world that scare. Here is a story for you.
Your chair.

\vspace{0.4cm}
\rightline{A.L.T.}

